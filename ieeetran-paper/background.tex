\section{Background} \label{sec-background}

This section makes a brief introduction to ProActive \cite{huet04high}, the basic
computing framework our solution extends.

\subsection{ProActive}

ProActive Parallel Suite developed by French national laboratory INRIA, is an innovative
open-source solution for acceleration of applications. It is seamlessly integrated with
the management of high-performance clouds, and simplifies the parallel program development
in cluster, grid and cloud computing environment. ProActive is completely based on Java
development for parallel distributed computing, and does not make any modification on Java
Virtual Machine (JVM), so it can run on any operating system which supports a standard
Java environment. With a ProActive platform, the users tackle the acceleration and
orchestration of all demanding applications easily \cite{ProActive}.

To execute parallel tasks in distributed environments, such as clouds and clusters,
requires a unified mechanism for scheduling tasks and managing resources. The original
ProActive is equipped with a batch scheduler \cite{ProActiveScheduling}. The scheduler
provides an abstraction of resources for users. It allows users to submit jobs which contain
one or several tasks, and to execute these tasks on available resources afterwards. It enables
several users to share the same resource pool and to tackle the problem involved with
distributed environment, such as failing tasks or resources. It also allows users to kill a
specified task easily, and to start the task again in another node.

By default, the scheduler assigns tasks according to the default FIFO (First In First Out)
priority policy. The user can increase the task's priority according to its emergency,
or can change the basic policy. In ProActive, to create and add a new scheduling
policy is very simple. The users only need to implement the policy interface, and execute
the scheduler with a new policy.

The ProActive Scheduler \cite{pascheduling} is connected to a Resource Manager \cite{parm}
which provides the resource abstraction. The Resource Manager is a component for resource
aggregation across the network. It sends compute units represented by ProActive nodes
(Java Virtual Machines runs the ProActive Runtime) to the scheduler which manages the
task workflow and distributes tasks to accessible resources. According to the deployment,
it can retrieve the computing nodes using different standard such as SSH, LSF, OAR, gLite,
EC2, etc.

With Scheduler, Resource Manager and other components, the ProActive platform can
 easily run on a cluster, a grid or a cloud or any mixture of them without
modification. High performance and distributed computing could perform better. 

\subsection{Outliers}

Many previous works have identified outliers as a critical performance killer in
heterogeneous clouds. Even in high end clusters, outliers are an inconvenient issue. Take our
experimental cluster for example. We share this large local cluster, which has 8800 CPU
cores over 740 server nodes, with many other institutes and researchers. As the local
cluster is mainly build for scientific computing, it only provides a Load Shared Facility
(LSF) job scheduling system for users to submit jobs in parallel. This situation is
similar to what happens in the cloud, where many virtual machines share some physical
servers.

Additionally, we have another small cluster that is dedicated to our experiment. To
better utilize all parts of computational resources, we deployed a uniform ProActive
Scheduling System to administrate all available resources. Based on the ProActive
platform, we successfully extend the flexibility of using large-scale clusters. In such a
heterogeneous environment, outliers often emerge.

The traditional ProActive global scheduler does not support collaboration and
communication with the local job scheduler on a cluster, so that too much CPU contention
on some node leads to outliers. The outliers prolong the completion time of job and seriously harm
the performance of ProActive.
