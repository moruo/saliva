\section{Background}


This section mainly illustrate the origin of this work and makes a brief introduction to ProActive, the basic computing framework our solution extends.

\subsection{What inspired this work?}

In our campus, we share a large local cluster having hundreds server nodes with many other institutes of research. As the local cluster is mainly build for scientific computing, it only provides a Load Shared Facility (LSF) job scheduling system for users submitting parallel or serial jobs. Additionally we have a small cluster we own. So we deployed a ProActive Scheduling System to administrate our computing resources uniformly. Benefited with ProActive platform, we extremely extends the flexibility of using the large cluster. But this introduce another problem that there are often a few outliers in a massive parallel job execution. Because of there is no collaboration and communication between the ProActive Scheduler and the job scheduler of the local cluster, computing resources contention leads to undetermined outliers sometimes. The outliers prolong the completion time of job and harm the performance of ProActive Scheduler seriously. This situation is similar to cloud, many virtual machines share some physical computational nodes. And lots of jobs executing on the ProActive Scheduler are scientific workflows and simulations, which are computing-intensive task. This problem inspires us to launch this work.

\subsection{ProActive}

ProActive Parallel Suite developed by French national laboratory INRIA, is an innovative open-source solution for acceleration of applications. It is seamlessly integrated with the management of high-performance clouds, and simplifies the parallel program development in cluster, grid and cloud computing environment. ProActive is completely based on Java development for parallel distributed computing, and does not make any modification on Java Virtual Machine (JVM), so it can run on any operating system which supports a standard Java environment. With a ProActive platform, the users tackle the acceleration and orchestration of all demanding applications easily. %\cite 

To execute parallel tasks in distributed environments, such as clouds and clusters, requires a unified mechanism for scheduling tasks and managing resources, and the original ProActive is equipped with a batch scheduler. %\cite
The scheduler provides an abstraction of resources for users. It allows users to submit jobs, containing one or several tasks, and then to execute these tasks on available resources. It enables several users to share the same resource pool and also to tackle the problem involved with distributed environment, like failing tasks or resources. It also allows users to kill a specified task easily, and start the task again in another node.

By default, the scheduler assign tasks according to the default FIFO (First In First Out) priority policy. If a task needs to be scheduled quickly, the user can increase its priority, or change such basic policy. In ProActive, to create and add a new scheduling policy is very simply. The users only need to implement the policy interface, and execute the scheduler with the new policy as argument.

The ProActive Scheduler is connected to a Resource Manager providing therefore the resource abstraction. %\cite
The Resource Manager is a component for resource aggregation across the network. It sends compute units represented by ProActive nodes (Java Virtual Machines running the ProActive Runtime)   to the scheduler which manages the task workflow and distributed tasks on accessible resources. According to the deployment, it can retrieve the computing nodes using different standard such as SSH, LSF, OAR, gLite, EC2, etc.

Including Scheduler, Resource Manager and other components, the ProActive platform can thus easily run on a cluster, a grid or a cloud or any mixture of them without modification, and get better performance for high performance and distributed computing.

